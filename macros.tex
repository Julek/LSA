
\newcommand{\figref}[1]{Fig.~\ref{#1}}
\newcommand{\Figref}[1]{Fig.~\ref{#1}}

\newcommand{\tautada}{Total-TaDA\xspace}

% Set
\newcommand{\Set}[1]{\left\{#1\right\}}
% Set builder
\newcommand{\Setb}[2]{\left\{#1 \ \middle| \ #2 \right\}}

\newcommand{\powset}[1]{\mathcal{P}( {#1} )}
\newcommand{\finpowset}[1]{\mathcal{P}_\mathit{fin}( {#1} )}
\newcommand{\ucpowset}[1]{\mathcal{P}^{\uparrow}( {#1} )}
\newcommand{\finmulset}[1]{\mathcal{M}_\mathit{fin}( {#1} )}

\newcommand{\eqdef}{\stackrel{\mathrm{def}}{=}}
\newcommand{\iffdef}{\stackrel{\mathrm{def}}{\iff}}

\newcommand{\pfun}{\rightharpoonup}
\newcommand{\fpfun}{\pfun_\mathit{fin}}
\newcommand{\tomon}{\to_\mathit{mon}}

\newcommand{\dom}{\mathop{\mathrm{dom}}}

\newcommand{\bigoast}{\mathop{\mbox{\Large $\varoast$}}}

\newcommand{\lcat}{\mathbin{\mathord{+}\!\mathord{+}}}

% Stability judgement
\newcommand{\stable}[2][]{{#1} \vDash {#2} \ \mathsf{stable}}

% A thread is a pair of variable store and command
\newcommand{\threadp}[2]{\langle {#1}, {#2} \rangle}

% Variable ranging over threads
\newcommand{\thread}{\tau}

% Command
\newcommand{\cmd}{\mathbb{C}}
% Boolean expression
\newcommand{\bexp}{\mathbb{B}}
% Expression
\newcommand{\vexp}{\mathbb{E}}

% Variable store
\newcommand{\sto}{s}

% Denotational semantics
\newcommand{\sem}[1]{\llbracket {#1} \rrbracket}
% Semantics of boolean expressions
\newcommand{\bsem}[2]{\mathcal{B} \sem{#1}_{#2}}
% Semantics of value expressions
\newcommand{\esem}[2]{\mathcal{E} \sem{#1}_{#2}}

% Transition label (operational)
\newcommand{\lab}{a}

% Identity transition label
\newcommand{\id}{\mathsf{id}}

\newcommand{\Level}{\mathsf{Level}}

% read transition label
\newcommand{\lread}{\mathsf{read}}
\newcommand{\lwrite}{\mathsf{write}}
\newcommand{\lcas}{\mathsf{cas}}
\newcommand{\lalloc}{\mathsf{alloc}}
\newcommand{\lspawn}{\mathsf{spawn}}

% Labelled transition relation
\newcommand{\red}[1][\id]{\xrightarrow{#1}}%{\stackrel{#1}{\to}}

% Programming language

\newcommand{\pvar}[1]{\mathtt{#1}}
\newcommand{\pfunction}[3]{\mathtt{function} \ #1(#2) \ \{ \ #3 \}}
\newcommand{\psequence}[2]{#1 ;\ #2}
\newcommand{\pifelse}[3]{\mathtt{if} \ (#1) \ #2 \ \mathtt{else} \ #3}
\newcommand{\pwhile}[2]{\mathtt{while} \ (#1) \ #2}
\newcommand{\pdowhile}[2]{\mathtt{do} \ #1 \ \mathtt{while} \ (#2)}
\newcommand{\plassign}[2]{#1 \mathtt{\ :=\ } #2}
\newcommand{\passign}[2]{[#1] \mathtt{\ :=\ } #2}
\newcommand{\pderef}[2]{#1 \mathtt{\ :=\ } [#2]}
\newcommand{\pfuncall}[3]{#1 \mathtt{\ :=\ } #2(#3)}
\newcommand{\preturn}[1]{\mathtt{return} \ #1}
\newcommand{\pskip}[0]{\mathtt{skip}}
\newcommand{\pfork}[2]{\mathtt{fork} \ #1(#2)} 
\newcommand{\pcas}[4]{#1 \mathtt{\ :=\ } \mathtt{CAS}(#2,#3,#4)}
\newcommand{\palloc}[2]{#1 \mathtt{\ :=\ } \mathtt{alloc}(#2)}

% Alternative programming languages to use in programs

\newcommand{\pfunctions}[2]{\mathtt{function} \ \mathtt{#1}(#2) \ \{}
\newcommand{\pfunctione}{\}}
\newcommand{\pifelses}[1]{\mathtt{if} \ (#1) \ \{}
\newcommand{\pifelsem}{\} \ \mathtt{else} \ \{}
\newcommand{\pifelsee}{\}}
\newcommand{\pwhiles}[1]{\mathtt{while} \ (#1) \ \{}
\newcommand{\pwhilee}{\}}
\newcommand{\pdowhiles}{\mathtt{do} \ \{}
\newcommand{\pdowhilee}[1]{\} \ \mathtt{while} \ (#1);}

\newcommand{\procenv}{\gamma}
\newcommand{\proccode}[1]{\mathit{code}(#1)}
\newcommand{\procvars}[1]{\mathit{vars}(#1)}

\newcommand{\procname}{f}

\newcommand{\threadpool}{T}
\newcommand{\tppar}{\mathrel{\|}}

\newcommand{\wop}{\mathbin{\cdot}}
\newcommand{\wemp}{\mathsf{emp}}

% Types

\newcommand{\Heap}{\mathsf{Heap}}
\newcommand{\World}{\mathsf{World}}
\newcommand{\RId}{\mathsf{RId}}
\newcommand{\Guard}{\mathsf{Guard}}
\newcommand{\GuardAlgebra}{\mathsf{GAlg}}
\newcommand{\AState}{\mathsf{AState}}
\newcommand{\ASTS}{\mathsf{ASTS}}
\newcommand{\ARType}{\mathsf{ARType}} % Abstract Region Type
\newcommand{\RTName}{\mathsf{RTName}} %Region type name
\newcommand{\Val}{\mathsf{Val}}
\newcommand{\Loc}{\mathsf{Loc}}
\newcommand{\APName}{\mathsf{APName}}
\newcommand{\APBag}{\mathsf{APBag}}
\newcommand{\RAss}{\mathsf{RAss}}
\newcommand{\GAss}{\mathsf{GAss}}
\newcommand{\RState}{\mathsf{RState}}
\newcommand{\WPred}{\mathsf{WPred}}
\newcommand{\AWorld}{\mathsf{AWorld}}
\newcommand{\AWPred}{\mathsf{AWPred}}
\newcommand{\AContext}{\mathsf{AContext}}
\newcommand{\View}{\mathsf{View}}
\newcommand{\RInterp}{\mathsf{RInterp}}
\newcommand{\APInterp}{\mathsf{APInterp}}
\newcommand{\AAction}{\mathsf{AAction}}
\newcommand{\FAction}{\mathsf{FAction}}

\newcommand{\Command}{\mathsf{Command}}
\newcommand{\Store}{\mathsf{Store}}

% variables
\newcommand{\ga}{\zeta} % guard algebra
\newcommand{\gcarr}{\mathcal{G}} % carrier set
\newcommand{\gzero}{\mathbf{0}} % identity
\newcommand{\gone}{\mathbf{1}} % maximal element
\newcommand{\gop}{\mathbin{\bullet}} % operator
\newcommand{\disjoint}{\mathrel{\#}}

\newcommand{\glts}{\mathcal{T}}

\newcommand{\art}{t} % Abstraction region type
\newcommand{\hp}{h} % Heap

\newcommand{\ap}{\mathsf{a}} % Abstract predicate?
\newcommand{\apb}{b} % Abstract predicate bag

\newcommand{\rass}{r}
\newcommand{\rset}{\mathcal{R}}
\newcommand{\aset}{\mathcal{A}}
\newcommand{\world}{w}
\newcommand{\rid}{a}
\newcommand{\gass}{\gamma}
\newcommand{\rstate}{\rho}
\newcommand{\aworld}{\varphi}
\newcommand{\acontext}{\mathcal{A}}
\newcommand{\donemap}{d}

\newcommand{\rrel}{\mathrel{\mathrm{R}}}
\newcommand{\grel}[1]{\mathrel{\mathrm{G}_{#1}}}

\newcommand{\rint}{I}
\newcommand{\apint}{\iota}

\newcommand{\levl}{\lambda}

% Projection functions
\newcommand{\regionpart}[1]{\rstate_{#1}}
\newcommand{\donepart}[1]{\donemap_{#1}}

% Atomic satisfaction judgement
\newcommand{\asat}[8]{#1 ; #2 \vDash \langle #4 \mid #5 \rangle  \ {#3} \  \langle #6 \mid - \rangle + \langle #8 \mid #7 \rangle }
\newcommand{\asatb}[5]{#1 ; #2 \vDash \left\langle #3 \middle\rangle \mathrel{{#4}} \middle\langle #5 \right\rangle}
\newcommand{\asatbv}[5]{#1 ; #2 \vDash \begin{array}{@{}c@{}}\left\langle #3 \right\rangle \\ \mathrel{{#4}} \\ \left\langle #5 \right\rangle \end{array}}

\newcommand{\tauasat}[8]{#1 ; #2 \vDash_\tau \langle #4 \mid #5 \rangle  \ {#3} \  \langle #6 \mid - \rangle + \langle #8 \mid #7 \rangle }
\newcommand{\tauasatb}[5]{#1 ; #2 \vDash_\tau \left\langle #3 \middle\rangle \mathrel{{#4}} \middle\langle #5 \right\rangle}
\newcommand{\tauasatbv}[5]{#1 ; #2 \vDash_\tau \begin{array}{@{}c@{}}\left\langle #3 \right\rangle \\ \mathrel{{#4}} \\ \left\langle #5 \right\rangle \end{array}}

\newcommand{\removedone}[2]{\mathrm{removedone}_{#1} \left({#2}\right)}

\newcommand{\region}[3]{\mathbf{#1}_{#2}(#3)}
\newcommand{\regionl}[4]{\mathbf{#1}_{#2}^{#4}(#3)}

\newcommand{\done}[2]{{#1} \Mapsto {#2}}

\newcommand{\rcol}[2]{{#1} \mathord{\downarrow_{#2}}}

\newcommand{\apcol}[2]{{#1} \mathord{\downharpoonright_{#2}}}
\newcommand{\apfcol}[1]{{#1} \mathord{\downharpoonright}}

\newcommand{\reify}[2]{ \lfloor {#1} \rfloor_{#2}} 

% atomic triple quantifier

\newcommand{\aall}{\rotatebox[origin=c]{180}{$\mathds{A}$}}

% existential...

\newcommand{\aexists}{\rotatebox[origin=c]{180}{$\mathds{E}$}}

% Atomicity Context Element

\newcommand{\aaction}[3]{{#1} : {\begin{array}{@{}c@{}}#2\end{array}} \rightsquigarrow {\begin{array}{@{}c@{}}#3\end{array}}}
\newcommand{\vaaction}[3]{\begin{array}{@{}l@{}}{#1} : {\begin{array}{@{}c@{}}#2\end{array}} \\ \qquad \rightsquigarrow {\begin{array}{@{}c@{}}#3\end{array}}\end{array}}

% triples

\newcommand{\apost}[2]{{\left\lgroup #1 \, \middle| \, #2 \right\rgroup}}

% Basic Hoare triple
% {#1} #2 {#3}
\newcommand{\triple}[3]{{\color{blue} \left\{ \begin{array}{@{}c@{}} #1 \end{array} \middle\} \ {\color{black}{#2}} \ \middle\{ \begin{array}{@{}c@{}}#3 \end{array} \right\}}}

% Vertical basic Hoare triple
% {#1}
%  #2
% {#3}
\newcommand{\vtriple}[3]{\begin{array}{@{}c@{}} {\color{blue}\left\{ {#1} \right\}} \\ {#2} \\ {\color{blue}\left\{ {#3} \right\}} \end{array}}

% Atomic Hoare triple
% #1 |- (AA) #2 . < #3 | #4 > #5 < #6 | #7 >
\newcommand{\atriple}[7]{ {\color{violet}#1} \vdash {\color{RoyalBlue} \aall #2 \ldotp} {\color{blue} \left\langle \begin{array}{@{}c@{}} #3 \end{array} \, \middle| \, \begin{array}{@{}c@{}} #4 \end{array} \middle\rangle \ {\color{black}#5} \ \middle\langle \begin{array}{@{}c@{}} #6 \end{array} \, \middle| \, \begin{array}{@{}c@{}} #7 \end{array} \right\rangle}}

% Atomic Hoare triple with exists
% #1 |- (AA) #2 . < #3 | #4 > #5 (EE) #6. < #7 | #8 >
\newcommand{\atriplee}[8]{ {\color{violet}#1} \vdash {\color{RoyalBlue} \aall #2 \ldotp} {\color{blue} \left\langle \begin{array}{@{}c@{}} #3 \end{array} \, \middle| \, \begin{array}{@{}c@{}} #4 \end{array} \middle\rangle \ {\color{black}#5} \quad {\color{RoyalBlue}\aexists #6 \ldotp} \middle\langle \begin{array}{@{}c@{}} #7 \end{array} \, \middle| \, \begin{array}{@{}c@{}} #8 \end{array} \right\rangle}}

% Non-atomic Hoare triple
% #1 |- {#2} #3 {#4}
\newcommand{\triplena}[4]{{\color{violet}#1} \vdash \triple{#2}{#3}{#4}}

% Atomic triple with alternative layout
% #1 |- (AA) #2 .
%   < #3 | #4 > #5 (EE) #6 . < #7 | #8 >
\newcommand{\atripleeaux}[8]{\begin{array}{@{}l@{}}{\color{violet}#1} \vdash {\color{RoyalBlue} \aall #2 \ldotp} \\ \quad {\color{blue} \left\langle \begin{array}{@{}c@{}} #3 \end{array} \, \middle| \, \begin{array}{@{}c@{}} #4 \end{array} \middle\rangle \ {\color{black}#5} \quad {\color{violet}\aexists #6 \ldotp} \middle\langle \begin{array}{@{}c@{}} #7 \end{array} \, \middle| \, \begin{array}{@{}c@{}} #8 \end{array} \right\rangle \end{array}}}

% Non-atomic triple with alternative layout
% #1 |-
% {#2} #3 {#4}
\newcommand{\triplenaaux}[4]{\begin{array}{@{}l@{}} {\color{violet}#1} \vdash \\ \quad \triple{#2}{#3}{#4}\end{array}}

% Vertical non-atomic triple
%       {#2}
% #1 |-  #3
%       {#4}
\newcommand{\vtriplena}[4]{{\color{violet} #1} \vdash \begin{array}{@{}c@{}} {\color{blue} \left\{ {#2} \right\} } \\ {#3} \\ {\color{blue} \left\{ {#4} \right\}} \end{array}}

% Atomic triple with no quantifier or local
% #1 |- <#2> #3 <#4>
\newcommand{\atriplenlnq}[4]{ {\color{violet}#1} \vdash {\color{blue}\left\langle \begin{array}{@{}c@{}}#2\end{array}  \middle\rangle \ {\color{black}#3} \ \middle\langle \begin{array}{@{}c@{}}#4\end{array} \right\rangle}}

% Atomic triple with no local
% #1 |- (AA) #2 . <#3> #4 <#5>
\newcommand{\atriplenl}[5]{ {\color{violet}#1} \vdash {\color{RoyalBlue}\aall #2 \ldotp} {\color{blue}\left\langle \begin{array}{@{}c@{}}#3\end{array} \middle\rangle \ {\color{black}#4} \ \middle\langle \begin{array}{@{}c@{}}#5\end{array} \right\rangle}}

% Atomic triple with no quantifiers
% #1 |- < #2 | #3 > #4 < #5 | #6 >
\newcommand{\atriplenq}[6]{ {\color{violet}#1} \vdash {\color{blue}\left\langle \begin{array}{@{}c@{}} #2 \end{array} \, \middle| \, \begin{array}{@{}c@{}} #3 \end{array} \middle\rangle \ {\color{black}#4} \ \middle\langle \begin{array}{@{}c@{}} #5 \end{array} \, \middle| \, \begin{array}{@{}c@{}} #6 \end{array} \right\rangle}}

% Vertical atomic triple
% #1 |- (AA) #2 . <#3 | #4>
%                     #5
%                 <#6 | #7>
\newcommand{\vatriple}[7]{\begin{array}{@{}l@{}c@{}} {\color{violet}#1} \vdash {\color{RoyalBlue}\aall #2 \ldotp} & {\color{blue}\left\langle \begin{array}{@{}c@{}} #3 \end{array} \, \middle| \, \begin{array}{@{}c@{}} #4 \end{array} \right\rangle} \\ & #5 \\ & {\color{blue}\left\langle \begin{array}{@{}c@{}} #6 \end{array} \, \middle| \, \begin{array}{@{}c@{}} #7 \end{array} \right\rangle} \end{array}}

% Vertical atomic triple with exists
% #1 |- (AA) #2 . <#3 | #4>
%                     #5
%       (EE) #6   <#7 | #8>
\newcommand{\vatriplee}[8]{{\color{violet}#1} \vdash \begin{array}{@{}c@{}} {\color{RoyalBlue}\aall #2 \ldotp} {\color{blue}\left\langle \begin{array}{@{}c@{}} #3 \end{array} \, \middle| \, \begin{array}{@{}c@{}} #4 \end{array} \right\rangle} \\ #5  \\ {\color{RoyalBlue}\aexists #6 \ldotp} {\color{blue}\left\langle \begin{array}{@{}c@{}} #7 \end{array} \, \middle| \, \begin{array}{@{}c@{}} #8 \end{array} \right\rangle} \end{array}}

% Vertical atomic triple with no local
%                 < #3 >
% #1 |- (AA) #2 .   #4
%                 < #5 >
\newcommand{\vatriplenl}[5]{\begin{array}{@{}l@{}c@{}} {\color{violet}#1} \vdash {\color{RoyalBlue}\aall #2 \ldotp} & {\color{blue}\left\langle \begin{array}{@{}c@{}}#3\end{array} \right\rangle} \\ & #4 \\ & {\color{blue}\left\langle \begin{array}{@{}c@{}}#5\end{array} \right\rangle} \end{array}}

% tau triples

% Atomic Hoare triple
% [#1] #2 |-t (AA) #3 . < #4 | #5 > #6 < #7 | #8 >
\newcommand{\tauatriple}[8][]{ #1 {\color{violet}#2} \vdash_\tau {\color{RoyalBlue} \aall #3 \ldotp} {\color{blue} \left\langle \begin{array}{@{}c@{}} #4 \end{array} \, \middle| \, \begin{array}{@{}c@{}} #5 \end{array} \middle\rangle \ {\color{black}#6} \ \middle\langle \begin{array}{@{}c@{}} #7 \end{array} \, \middle| \, \begin{array}{@{}c@{}} #8 \end{array} \right\rangle}}

% Atomic Hoare triple with exists
% [#1] #2 |-t (AA) #3 . < #4 | #5 > #6 (EE) #7. < #8 | #9 >
\newcommand{\tauatriplee}[9][]{ #1 {\color{violet}#2} \vdash_\tau {\color{RoyalBlue} \aall #3 \ldotp} {\color{blue} \left\langle \begin{array}{@{}c@{}} #4 \end{array} \, \middle| \, \begin{array}{@{}c@{}} #5 \end{array} \middle\rangle \ {\color{black}#6} \quad {\color{RoyalBlue}\aexists #7 \ldotp} \middle\langle \begin{array}{@{}c@{}} #8 \end{array} \, \middle| \, \begin{array}{@{}c@{}} #9 \end{array} \right\rangle}}

% Non-atomic Hoare triple
% [#1] #2 |-t {#3} #4 {#5}
\newcommand{\tautriplena}[5][]{#1 {\color{violet}#2} \vdash_\tau \triple{#3}{#4}{#5}}

% Atomic triple with alternative layout
% [#1] #2 |-t (AA) #3 .
%   < #4 | #5 > #6 (EE) #7 . < #8 | #9 >
\newcommand{\tauatripleeaux}[9][]{\begin{array}{@{}l@{}} #1 {\color{violet}#2} \vdash_\tau {\color{RoyalBlue} \aall #3 \ldotp} \\ \quad {\color{blue} \left\langle \begin{array}{@{}c@{}} #4 \end{array} \, \middle| \, \begin{array}{@{}c@{}} #5 \end{array} \middle\rangle \ {\color{black}#6} \quad {\color{violet}\aexists #7 \ldotp} \middle\langle \begin{array}{@{}c@{}} #8 \end{array} \, \middle| \, \begin{array}{@{}c@{}} #9 \end{array} \right\rangle \end{array}}}

% Non-atomic triple with alternative layout
% [#1] #2 |-t
% {#3} #4 {#5}
\newcommand{\tautriplenaaux}[5][]{\begin{array}{@{}l@{}} #1 {\color{violet}#2} \vdash_\tau \\ \quad \triple{#3}{#4}{#5}\end{array}}

% Vertical non-atomic triple
%       {#2}
% #1 |-t #3
%       {#4}
\newcommand{\vtautriplena}[4]{{\color{violet} #1} \vdash_\tau \begin{array}{@{}c@{}} {\color{blue} \left\{ {#2} \right\} } \\ {#3} \\ {\color{blue} \left\{ {#4} \right\}} \end{array}}

% Atomic triple with no quantifier or local
% #1 |-t <#2> #3 <#4>
\newcommand{\tauatriplenlnq}[4]{ {\color{violet}#1} \vdash_\tau {\color{blue}\left\langle \begin{array}{@{}c@{}}#2\end{array}  \middle\rangle \ {\color{black}#3} \ \middle\langle \begin{array}{@{}c@{}}#4\end{array} \right\rangle}}

% Atomic triple with no local
% [#1] #2 |-t (AA) #3 . <#4> #5 <#6>
\newcommand{\tauatriplenl}[6][]{ #1 {\color{violet}#2} \vdash_\tau {\color{RoyalBlue}\aall #3 \ldotp} {\color{blue}\left\langle \begin{array}{@{}c@{}}#4\end{array} \middle\rangle \ {\color{black}#5} \ \middle\langle \begin{array}{@{}c@{}}#6\end{array} \right\rangle}}

% Atomic triple with no local with alternative layout
% [#1] #2 |-t
% (AA) #3 . <#4> #5 <#6>
\newcommand{\tauatriplenlaux}[6][]{\begin{array}{@{}l@{}} #1 {\color{violet}#2} \vdash_\tau \\ \quad {\color{RoyalBlue}\aall #3 \ldotp} {\color{blue}\left\langle \begin{array}{@{}c@{}}#4\end{array} \middle\rangle \ {\color{black}#5} \ \middle\langle \begin{array}{@{}c@{}}#6\end{array} \right\rangle}\end{array}}

% Atomic triple with no quantifiers
% #1 |-t < #2 | #3 > #4 < #5 | #6 >
\newcommand{\tauatriplenq}[6]{ {\color{violet}#1} \vdash_\tau {\color{blue}\left\langle \begin{array}{@{}c@{}} #2 \end{array} \, \middle| \, \begin{array}{@{}c@{}} #3 \end{array} \middle\rangle \ {\color{black}#4} \ \middle\langle \begin{array}{@{}c@{}} #5 \end{array} \, \middle| \, \begin{array}{@{}c@{}} #6 \end{array} \right\rangle}}

% Vertical atomic triple
% #1 |-t (AA) #2 . <#3 | #4>
%                      #5
%                  <#6 | #7>
\newcommand{\vtauatriple}[7]{\begin{array}{@{}l@{}c@{}} {\color{violet}#1} \vdash_\tau {\color{RoyalBlue}\aall #2 \ldotp} & {\color{blue}\left\langle \begin{array}{@{}c@{}} #3 \end{array} \, \middle| \, \begin{array}{@{}c@{}} #4 \end{array} \right\rangle} \\ & #5 \\ & {\color{blue}\left\langle \begin{array}{@{}c@{}} #6 \end{array} \, \middle| \, \begin{array}{@{}c@{}} #7 \end{array} \right\rangle} \end{array}}

% Vertical atomic triple with exists
% [#1] #2 |-t (AA) #3 . <#4 | #5>
%                                     #6
%                 (EE) #7   <#8 | #9>
\newcommand{\vtauatriplee}[9][]{{#1 \color{violet}#2} \vdash_\tau \begin{array}{@{}c@{}} {\color{RoyalBlue}\aall #3 \ldotp} {\color{blue}\left\langle \begin{array}{@{}c@{}} #4 \end{array} \, \middle| \, \begin{array}{@{}c@{}} #5 \end{array} \right\rangle} \\ #6  \\ {\color{RoyalBlue}\aexists #7 \ldotp} {\color{blue}\left\langle \begin{array}{@{}c@{}} #8 \end{array} \, \middle| \, \begin{array}{@{}c@{}} #9 \end{array} \right\rangle} \end{array}}

% Vertical atomic triple with no local
%                               < #4 >
% [#1] #2 |-t (AA) #3 .  #5
%                              < #6 >
\newcommand{\vtauatriplenl}[6][]{\begin{array}{@{}l@{}c@{}} #1 {\color{violet}#2} \vdash_\tau {\color{RoyalBlue}\aall #3 \ldotp} & {\color{blue}\left\langle \begin{array}{@{}c@{}}#4\end{array} \right\rangle} \\ & #5 \\ & {\color{blue}\left\langle \begin{array}{@{}c@{}}#6\end{array} \right\rangle} \end{array}}



\newcommand{\aspecline}[1]{{\color{blue}\left\langle\begin{array}{@{}l@{}} #1 \end{array}\right\rangle}}
\newcommand{\aspeclinep}[2]{{\color{blue}\left\langle\begin{array}{@{}l@{}} #1 \end{array}\, \middle| \, \begin{array}{l} #2 \end{array}\right\rangle}}
\newcommand{\specline}[1]{{\color{blue}\left\{\begin{array}{@{}l@{}} #1 \end{array}\right\}}}
\newcommand{\acontextline}[1]{{\color{violet}#1}}
\newcommand{\quantifierline}[1]{{\color{RoyalBlue}#1}}

% An environment for annotating proof outlines
\newenvironment{leftvruled}[2][10pt]{\begin{array}{@{}m{#1}|l@{}}\text{\rotatebox{90}{\ {#2}\ }}&\begin{array}{@{}l@{}}}{\end{array}\end{array}}

\newcommand{\ite}[3]{\underline{\mathbf{if}} \ {#1} \ \underline{\mathbf{then}} \ {#2} \ \underline{\mathbf{else}} \ {#3}}
\newcommand{\itel}[3]{\underline{\mathbf{if}} \ {#1} \ \underline{\mathbf{then}} \ {#2} \\ \quad \underline{\mathbf{else}} \ {#3}}
\newcommand{\itell}[3]{\underline{\mathbf{if}} \ {#1} \\ \underline{\mathbf{then}} \ {#2} \\ \underline{\mathbf{else}} \ {#3}}

% RGSep / CAP
\newcommand{\shared}[1]{{\fboxsep=0.4pt\fbox{\ensuremath{\;\begin{array}{@{}r@{}}#1\end{array}\;}}\hspace{1pt}}}
\newcommand{\Fshare}[3]{\shared{#1}^{#2}_{#3}}
\newcommand{\shareds}[1]{{\fboxsep=1pt\fbox{\ensuremath{{\phantom{||}\!\!\!}#1\,}}\hspace{1pt}}}
\newcommand{\Fshares}[3]{\shareds{#1}^{#2}_{#3}}

% Length-of
\newcommand{\lo}[1]{\left| {#1} \right|}

